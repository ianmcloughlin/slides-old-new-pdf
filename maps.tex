\documentclass{beamer}

\usepackage{iansslides}

\newenvironment{novspace}{\abovedisplayskip=0pt \belowdisplayskip=0pt \abovedisplayshortskip=0pt \belowdisplayshortskip=0pt}{}

\newenvironment{topdisp}{\color{gmitred}\vspace{-4mm}}{\vspace{-4mm}}

\begin{document}

\section{Maps}

\begin{frame}{Maps}

  \begin{alertblock}{Definition of map}
    A map from a set $A$ to a set $B$ is a subset $M$ of $A \times B$ where each element of $A$ appears as the first element of a tuple in $M$ exactly once.
  \end{alertblock}

  \begin{topdisp}
    $$A = \{a,b,c\} \qquad B = \{x,y,z\}$$
  \end{topdisp}

  \begin{minipage}[t]{0.48\linewidth}
  \begin{exampleblock}{Maps}
    \begin{itemize}
      \item $\{(a,x),(b,x),(c,x)\}$
      \item $\{(a,x),(b,y),(c,z)\}$
    \end{itemize}
  \end{exampleblock}
\end{minipage}
\begin{minipage}[t]{0.48\linewidth}
  \begin{exampleblock}{Not maps}
    \begin{itemize}
      \item $\{(a,x),(a,y),(b,x),(c,x)\}$
      \item $\{(a,x),(b,y)\}$
    \end{itemize}
  \end{exampleblock}
\end{minipage}
\end{frame}


\begin{frame}{One-to-one map}
  \begin{topdisp}
    $$ A = \{1,2,3\} \quad  B = \{a,b,c,d\} \quad  M = \{ (1,a), (2,b), (3,d) \} $$
  \end{topdisp}
  \begin{itemize}
    \setlength\itemsep{3mm}
    \item In a map, $M \subseteq A \times B$, two or more distinct elements of $A$ can be mapped to the same element of $B$.
    \item A map where this does not happen is described as \textbf{one-to-one}.
    \item So, a map in which distinct elements of $A$ go to distinct elements of $B$ is one-to-one.
  \end{itemize}
\end{frame}

\begin{frame}{Onto map}
  \begin{topdisp}
    $$ A = \{1,2,3\} \quad  B = \{a,b,c,d\} \quad  M = \{ (1,a), (2,a), (3,b) \} $$
  \end{topdisp}
  \begin{itemize}
    \setlength\itemsep{3mm}
    \item In a map, not all of the elements of $B$ need to be involved in the map.
    \item A map in which they are all involved is described as \textbf{onto}.
    \item So, a map in which each element of $B$ is paired with an element of $A$ is onto.
  \end{itemize}
\end{frame}

\begin{frame}{Bijections}
  \begin{topdisp}
    $$ A = \{1,2,3\} \quad  B = \{a,b,c\} \quad  M = \{ (1,a), (2,c), (3,b) \} $$
  \end{topdisp}
  \begin{itemize}
    \setlength\itemsep{3mm}
    \item Maps can be neither one-to-one nor onto, one or the other, or both.
    \item A \textbf{bijection} is map that is both one-to-one and onto.
    \item Both $A$ and $B$ must have the same size in a bijection.
  \end{itemize}
\end{frame}


\begin{frame}{Partial maps}
  \begin{topdisp}
    $$ A = \{1,2,3\} \quad  B = \{a,b,c\} \quad  P = \{ (1,a), (3,c) \} $$
  \end{topdisp}
  \begin{itemize}
    \setlength\itemsep{3mm}
    \item A map $M$ from a set $A$ to a set $B$ must involve every element of $A$.
    \item A partial map is like a map, but with that condition relaxed.
    \item A partial map from a set $A$ to a set $B$ is a subset of $A \times B$ where any element of $A$ that appears as the first element in a tuple does so exactly once.
    \item The term \emph{partial map} is a bit of a misnomer, as a partial map is not necessarily a map, but a map is a partial map.
  \end{itemize}
\end{frame}

\end{document}