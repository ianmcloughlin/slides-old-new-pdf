\documentclass{beamer}

% Most of the preamble is in here.
\usepackage{iansslides}

\begin{document}


\section{Learning}


\begin{frame}{Memorize these symbols}
  \vspace{8mm}
  \centering
  \begin{adjustbox}{width=0.8\columnwidth, keepaspectratio}
    \begin{tikzpicture}
      \draw[thick,                color=gmitblue] (0,0) -- (1,0) -- (1,1);
      \draw[thick,shift={(0,-2)} ,color=gmitblue] (0,1) -- (0,0) -- (1,0) -- (1,1);
      \draw[thick,shift={(0,-4)} ,color=gmitblue] (0,1) -- (0,0) -- (1,0);
      \draw[thick,shift={(0,-6)} ,color=gmitblue] (0,1) -- (1,1) -- (1,0) -- (0,0);
      \draw[thick,shift={(4, 0)} ,color=gmitblue] (0,0) -- (0,1) -- (1,1) -- (1,0) -- (0,0);
      \draw[thick,shift={(4,-2)} ,color=gmitblue] (1,1) -- (0,1) -- (0,0) -- (1,0);
      \draw[thick,shift={(4,-4)} ,color=gmitblue] (0,1) -- (1,1) -- (1,0);
      \draw[thick,shift={(4,-6)} ,color=gmitblue] (0,0) -- (0,1) -- (1,1) -- (1,0);
      \draw[thick,shift={(8, 0)} ,color=gmitblue] (0,0) -- (0,1) -- (1,1);
      \node at (-0.5, 0.5) {\Large $1.$};
      \node at (-0.5,-1.5) {\Large $2.$};
      \node at (-0.5,-3.5) {\Large $3.$};
      \node at (-0.5,-5.5) {\Large $4.$};
      \node at ( 3.5, 0.5) {\Large $5.$};
      \node at ( 3.5,-1.5) {\Large $6.$};
      \node at ( 3.5,-3.5) {\Large $7.$};
      \node at ( 3.5,-5.5) {\Large $8.$};
      \node at ( 7.5, 0.5) {\Large $9.$};
    \end{tikzpicture}
  \end{adjustbox}
\end{frame}


\begin{frame}[standout]
  \begin{quote}
    Write down the symbols corresponding to the number 314,159.
  \end{quote}
\end{frame}


\begin{frame}
  \vspace{8mm}
  \centering
  %\begin{adjustbox}{width=0.8\columnwidth, keepaspectratio}
    \begin{tikzpicture}
      \draw[thick, color=gmitblue] (-2, 0) -- (4,0);
      \draw[thick, color=gmitblue] (-2, 2) -- (4,2);
      \draw[thick, color=gmitblue] ( 0,-2) -- (0,4);
      \draw[thick, color=gmitblue] ( 2,-2) -- (2,4);
      \node at (-1, 3) {\Large $1$};
      \node at ( 1, 3) {\Large $2$};
      \node at ( 3, 3) {\Large $3$};
      \node at (-1, 1) {\Large $4$};
      \node at ( 1, 1) {\Large $5$};
      \node at ( 3, 1) {\Large $6$};
      \node at (-1,-1) {\Large $7$};
      \node at ( 1,-1) {\Large $8$};
      \node at ( 3,-1) {\Large $9$};
    \end{tikzpicture}
  %\end{adjustbox}
\end{frame}


\begin{frame}[standout]
  \begin{quote}
    Knowledge is constructed as a result of the learner's activity. \\[8mm]
  \end{quote}

  {\scriptsize -- Teaching Teaching \& Understanding Understanding by Claus Brabrand}

\end{frame}


\begin{frame}{Zone of proximal development}
  \vspace{4mm}
  \centering
  \begin{adjustbox}{width=0.7\columnwidth, keepaspectratio}
    \begin{tikzpicture}
      \draw[fill=gmitgrey!50] (0,0) circle (6);
      \draw[fill=gmitred!50] (0,0) circle (4);
      \draw[fill=gmitblue!50] (0,0) circle (3);
      \node at (0,0) {\Large Know};
      \node at (0,-3.5) {\Large Learn};
      \node at (0,-5) {\Large Difficult};
    \end{tikzpicture}
  \end{adjustbox}
\end{frame} 


\end{document}