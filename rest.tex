\documentclass{beamer}

\usepackage{iansslides}

\begin{document}

\section{REST}


\begin{frame}{HTTP APIs}
	\begin{itemize}
		\item Facebook, Google, Reddit and others often provide programmable interfaces to their services.
		\item This lets other application developers use the services programmatically.
		\item For instance, Reddit allows developers to create mobile apps for viewing and making submissions to reddit.
		\item HTTP is often the mechanism used for this purpose.
		\item Access is provided through a set of URLs, across a variety of HTTP methods.
		\item The APIs often require JSON in HTTP request bodies and often return the query results as JSON.
	\end{itemize}
\end{frame}


\begin{frame}{REST}
  \begin{itemize}
    \item REST stands for Representational State Transfer.
    \item REST is an architecture describing how we might use HTTP.
    \item RESTful APIs make use of more HTTP methods than just GET and POST.
    \item Most HTTP APIs are not RESTful.
    \item RESTful APIs adhere to a few loosely defined constraints.
    \item Two of those constraints are that the API is stateless and cacheable.
  \end{itemize}
  \citeurl{drdobbs.com/web-development/restful-web-services-a-tutorial/240169069}
\end{frame}


\begin{frame}{HTTP and CRUD}
  \tikzset{->-/.style={decoration={markings,mark=at position .5 with {\arrow{>}}},postaction={decorate}}}
  \tikzset{-<-/.style={decoration={markings,mark=at position .5 with {\arrow{<}}},postaction={decorate}}}
  \tikzstyle{rect} = [rectangle,fill=gmitblue,text width=5em,text centered,minimum height=16em,rounded corners,text=white]
  \begin{adjustbox}{max width={0.9\textwidth},center} 
    \begin{tikzpicture}[thick]

      \node [rect] (database) at (10,0) {Database};
      \node [rect] (webserver) at (5,0) {Web \\ Server};
      \node [rect] (client) at (0,0) {Client};

      \pgfmathsetmacro{\levela}{6em}
      \pgfmathsetmacro{\levelb}{2em}
      \pgfmathsetmacro{\levelc}{-2em}
      \pgfmathsetmacro{\leveld}{-6em}
      \pgfmathsetmacro{\diff}{0.3em}

      \path ([yshift={\levela+\diff}]webserver.west) edge[<-] node[anchor=south] {\footnotesize POST}     ([yshift={\levela+\diff}]client.east);
      \path ([yshift={\levela-\diff}]webserver.west) edge[->]                                             ([yshift={\levela-\diff}]client.east);
      
      \path ([yshift={\levela+\diff}]database.west) edge[<-] node[anchor=south] {\footnotesize CREATE}    ([yshift={\levela+\diff}]webserver.east);
      \path ([yshift={\levela-\diff}]database.west) edge[->]                                              ([yshift={\levela-\diff}]webserver.east);

      \path ([yshift={\levelb+\diff}]webserver.west) edge[<-] node[anchor=south] {\footnotesize GET}      ([yshift={\levelb+\diff}]client.east);
      \path ([yshift={\levelb-\diff}]webserver.west) edge[->]                                             ([yshift={\levelb-\diff}]client.east);
      
      \path ([yshift={\levelb+\diff}]database.west) edge[<-] node[anchor=south] {\footnotesize RETRIEVE}  ([yshift={\levelb+\diff}]webserver.east);
      \path ([yshift={\levelb-\diff}]database.west) edge[->]                                              ([yshift={\levelb-\diff}]webserver.east);

      \path ([yshift={\levelc+\diff}]webserver.west) edge[<-] node[anchor=south] {\footnotesize PUT}      ([yshift={\levelc+\diff}]client.east);
      \path ([yshift={\levelc-\diff}]webserver.west) edge[->]                                             ([yshift={\levelc-\diff}]client.east);
      
      \path ([yshift={\levelc+\diff}]database.west) edge[<-] node[anchor=south] {\footnotesize UPDATE}    ([yshift={\levelc+\diff}]webserver.east);
      \path ([yshift={\levelc-\diff}]database.west) edge[->]                                              ([yshift={\levelc-\diff}]webserver.east);

      \path ([yshift={\leveld+\diff}]webserver.west) edge[<-] node[anchor=south] {\footnotesize DELETE}   ([yshift={\leveld+\diff}]client.east);
      \path ([yshift={\leveld-\diff}]webserver.west) edge[->]                                             ([yshift={\leveld-\diff}]client.east);
      
      \path ([yshift={\leveld+\diff}]database.west) edge[<-] node[anchor=south] {\footnotesize DELETE}  ([yshift={\leveld+\diff}]webserver.east);
      \path ([yshift={\leveld-\diff}]database.west) edge[->]                                              ([yshift={\leveld-\diff}]webserver.east);

      \draw ([yshift={\levela+\diff}]webserver.west)  edge[dashed,draw=gray!60,->-] ([yshift={\levela+\diff}]webserver.east);
      \draw ([yshift={\levela-\diff}]webserver.west)  edge[dashed,draw=gray!60,-<-] ([yshift={\levela-\diff}]webserver.east);
      \draw ([yshift={\levelb+\diff}]webserver.west)  edge[dashed,draw=gray!60,->-] ([yshift={\levelb+\diff}]webserver.east);
      \draw ([yshift={\levelb-\diff}]webserver.west)  edge[dashed,draw=gray!60,-<-] ([yshift={\levelb-\diff}]webserver.east);
      \draw ([yshift={\levelc+\diff}]webserver.west)  edge[dashed,draw=gray!60,->-] ([yshift={\levelc+\diff}]webserver.east);
      \draw ([yshift={\levelc-\diff}]webserver.west)  edge[dashed,draw=gray!60,-<-] ([yshift={\levelc-\diff}]webserver.east);
      \draw ([yshift={\leveld+\diff}]webserver.west)  edge[dashed,draw=gray!60,->-] ([yshift={\leveld+\diff}]webserver.east);
      \draw ([yshift={\leveld-\diff}]webserver.west)  edge[dashed,draw=gray!60,-<-] ([yshift={\leveld-\diff}]webserver.east);

      \draw[dashed,draw=gray!60] ([yshift={\levela+\diff}]database.west)  -- ([yshift={\levela+\diff}]database.center) -- ([yshift={\levela-\diff}]database.center) -- ([yshift={\levela-\diff}]database.west);
      \draw[dashed,draw=gray!60] ([yshift={\levelb+\diff}]database.west)  -- ([yshift={\levelb+\diff}]database.center) -- ([yshift={\levelb-\diff}]database.center) -- ([yshift={\levelb-\diff}]database.west);
      \draw[dashed,draw=gray!60] ([yshift={\levelc+\diff}]database.west)  -- ([yshift={\levelc+\diff}]database.center) -- ([yshift={\levelc-\diff}]database.center) -- ([yshift={\levelc-\diff}]database.west);
      \draw[dashed,draw=gray!60] ([yshift={\leveld+\diff}]database.west)  -- ([yshift={\leveld+\diff}]database.center) -- ([yshift={\leveld-\diff}]database.center) -- ([yshift={\leveld-\diff}]database.west);


    \end{tikzpicture}
  \end{adjustbox}
\end{frame}

\begin{frame}{AMP}
  \tikzset{->-/.style={decoration={markings,mark=at position .5 with {\arrow{>}}},postaction={decorate}}}
  \tikzset{-<-/.style={decoration={markings,mark=at position .5 with {\arrow{<}}},postaction={decorate}}}
  \tikzstyle{rect} = [rectangle,fill=gmitblue,text width=5em,text centered,minimum height=16em,rounded corners,text=white]
  \begin{adjustbox}{max width={0.9\textwidth},center} 
    \begin{tikzpicture}[thick]

      \node [rect] (database) at (10,0) {MySQL};
      \node [rect] (webserver) at (5,0) {Apache \\ and PHP};
      \node [rect] (client) at (0,0) {Firefox};

      \pgfmathsetmacro{\levela}{6em}
      \pgfmathsetmacro{\levelb}{2em}
      \pgfmathsetmacro{\levelc}{-2em}
      \pgfmathsetmacro{\leveld}{-6em}
      \pgfmathsetmacro{\diff}{0.3em}

      \path ([yshift={\levela+\diff}]webserver.west) edge[<-] node[anchor=south] {\footnotesize POST}     ([yshift={\levela+\diff}]client.east);
      \path ([yshift={\levela-\diff}]webserver.west) edge[->]                                             ([yshift={\levela-\diff}]client.east);
      
      \path ([yshift={\levela+\diff}]database.west) edge[<-] node[anchor=south] {\footnotesize INSERT}    ([yshift={\levela+\diff}]webserver.east);
      \path ([yshift={\levela-\diff}]database.west) edge[->]                                              ([yshift={\levela-\diff}]webserver.east);

      \path ([yshift={\levelb+\diff}]webserver.west) edge[<-] node[anchor=south] {\footnotesize GET}      ([yshift={\levelb+\diff}]client.east);
      \path ([yshift={\levelb-\diff}]webserver.west) edge[->]                                             ([yshift={\levelb-\diff}]client.east);
      
      \path ([yshift={\levelb+\diff}]database.west) edge[<-] node[anchor=south] {\footnotesize SELECT}  ([yshift={\levelb+\diff}]webserver.east);
      \path ([yshift={\levelb-\diff}]database.west) edge[->]                                              ([yshift={\levelb-\diff}]webserver.east);

      \path ([yshift={\levelc+\diff}]webserver.west) edge[<-] node[anchor=south] {\footnotesize PUT}      ([yshift={\levelc+\diff}]client.east);
      \path ([yshift={\levelc-\diff}]webserver.west) edge[->]                                             ([yshift={\levelc-\diff}]client.east);
      
      \path ([yshift={\levelc+\diff}]database.west) edge[<-] node[anchor=south] {\footnotesize UPDATE}    ([yshift={\levelc+\diff}]webserver.east);
      \path ([yshift={\levelc-\diff}]database.west) edge[->]                                              ([yshift={\levelc-\diff}]webserver.east);

      \path ([yshift={\leveld+\diff}]webserver.west) edge[<-] node[anchor=south] {\footnotesize DELETE}   ([yshift={\leveld+\diff}]client.east);
      \path ([yshift={\leveld-\diff}]webserver.west) edge[->]                                             ([yshift={\leveld-\diff}]client.east);
      
      \path ([yshift={\leveld+\diff}]database.west) edge[<-] node[anchor=south] {\footnotesize DELETE}  ([yshift={\leveld+\diff}]webserver.east);
      \path ([yshift={\leveld-\diff}]database.west) edge[->]                                              ([yshift={\leveld-\diff}]webserver.east);

      \draw ([yshift={\levela+\diff}]webserver.west)  edge[dashed,draw=gray!60,->-] ([yshift={\levela+\diff}]webserver.east);
      \draw ([yshift={\levela-\diff}]webserver.west)  edge[dashed,draw=gray!60,-<-] ([yshift={\levela-\diff}]webserver.east);
      \draw ([yshift={\levelb+\diff}]webserver.west)  edge[dashed,draw=gray!60,->-] ([yshift={\levelb+\diff}]webserver.east);
      \draw ([yshift={\levelb-\diff}]webserver.west)  edge[dashed,draw=gray!60,-<-] ([yshift={\levelb-\diff}]webserver.east);
      \draw ([yshift={\levelc+\diff}]webserver.west)  edge[dashed,draw=gray!60,->-] ([yshift={\levelc+\diff}]webserver.east);
      \draw ([yshift={\levelc-\diff}]webserver.west)  edge[dashed,draw=gray!60,-<-] ([yshift={\levelc-\diff}]webserver.east);
      \draw ([yshift={\leveld+\diff}]webserver.west)  edge[dashed,draw=gray!60,->-] ([yshift={\leveld+\diff}]webserver.east);
      \draw ([yshift={\leveld-\diff}]webserver.west)  edge[dashed,draw=gray!60,-<-] ([yshift={\leveld-\diff}]webserver.east);

      \draw[dashed,draw=gray!60] ([yshift={\levela+\diff}]database.west)  -- ([yshift={\levela+\diff}]database.center) -- ([yshift={\levela-\diff}]database.center) -- ([yshift={\levela-\diff}]database.west);
      \draw[dashed,draw=gray!60] ([yshift={\levelb+\diff}]database.west)  -- ([yshift={\levelb+\diff}]database.center) -- ([yshift={\levelb-\diff}]database.center) -- ([yshift={\levelb-\diff}]database.west);
      \draw[dashed,draw=gray!60] ([yshift={\levelc+\diff}]database.west)  -- ([yshift={\levelc+\diff}]database.center) -- ([yshift={\levelc-\diff}]database.center) -- ([yshift={\levelc-\diff}]database.west);
      \draw[dashed,draw=gray!60] ([yshift={\leveld+\diff}]database.west)  -- ([yshift={\leveld+\diff}]database.center) -- ([yshift={\leveld-\diff}]database.center) -- ([yshift={\leveld-\diff}]database.west);


    \end{tikzpicture}
  \end{adjustbox}
\end{frame}



\begin{frame}{Typical example}
  Suppose we have a system for storing and retrieving emails.
  \begin{table}
    \begin{tabular}{r@{\hspace{0.5cm}}|l@{\hspace{0.5cm}}|l}
      Method & URL & Description \\
      \hline
      GET    & /emails   & list all emails \\
      POST   & /email    & store new email \\
      GET    & /email/32 & retrieve email with id 32 \\
      PUT    & /email/32 & update email with id 32 \\
      DELETE & /email/32 & delete email with id 32
    \end{tabular}
  \end{table}
  \citeurl{bitworking.org/news/201/restify-daytrader}
\end{frame}


\begin{frame}{Stateless}
  \begin{itemize}
    \item Statelessness is a REST constraint.
    \item HTTP uses the client-server model.
    \item The server should treat each request as a single, independent transaction.
    \item No client state should be stored on the server.
    \item Each request must contain all of the information to perform the request. 
  \end{itemize}
\end{frame}


\begin{frame}{Cacheable}
  \begin{itemize}
    \item REST APIs should provide responses that are cacheable.
    \item Intermediaries between the client and server should be able to cache responses.
    \item This should be transparent to the client.
    \item Cacheability increases response time.
    \item Browsers usually cache resources, in case they are requested again.
    \item There is usually a time limit on cached resources.
  \end{itemize}
\end{frame}


\begin{frame}{CouchDB}
	\begin{itemize}
		\item CouchDB is a document-oriented database.
		\item Documents are represented in CouchDB as JSON objects.
		\item Each document has its own id and revision, indicated by properties \mintinline{js}{_id} and \mintinline{js}{_rev} in the JSON document.
		\item Updating a document leaves its \mintinline{js}{_id} intact, but updates its \mintinline{js}{_rev}.
		\item Different documents can have different properties -- there is no schema.
		\item The main interface with CouchDB, for storage and retrieval is a HTTP API.
		\item CouchDB uses HTTP methods such as \mintinline{http}{GET}, \mintinline{http}{POST}, \mintinline{http}{PUT} and \mintinline{http}{DELETE} to retrieve, add, update and delete documents.
	\end{itemize}
\end{frame}


\begin{frame}{NoSQL}
	\begin{itemize}
		\item NoSQL is the umbrella term for databases that do not conform to the relational, SQL-style model.
		\item Relational databases are good for some types of data.
		\item However, they have some issues.
		\item SQL queries can result in costly joins.
		\item Tables can be sparsely populated.
		\item Two common NoSQL database types are Document-oriented and Graph.
	\end{itemize}
\end{frame}


\begin{frame}{Futon}
	\begin{itemize}
		\item CouchDB has an in-built admin interface.
		\item It's called Futon.
		\item You access it through the \mintinline{html}{/_utils} path.
		\item You can create and delete databases.
		\item You can also create, update and delete documents.
	\end{itemize}
\end{frame}


\begin{frame}{MapReduce}
	\begin{itemize}
		\item MapReduce is a way of programming.
		\item It is a model for performing specific types of problems that are common in programming.
		\item MapReduce promotes algorithms that have an initially embarrassingly parallel part, and a subsequent consolidation part.
		\item The former is the Map part, and the latter is the Reduce part.
		\item MapReduce isn't necessarily anything new, the ideas have existed for a long time.
		\item The formalisation of those ideas and their implementation in systems such as Hadoop is useful.
	\end{itemize}
\citeurl{joelonsoftware.com/items/2006/08/01.html}
\end{frame}


\begin{frame}[fragile]{Map}
Map takes a function and a list, and applies the function to every element of the list.
  \begin{minted}{javascript}
function map(fn, a) {
  r = [];
  for (i = 0; i < a.length; i++)
  	r[i] = fn(a[i]);
	return r;
}
	\end{minted}
	\citeurl{joelonsoftware.com/items/2006/08/01.html}
\end{frame}


\begin{frame}[fragile]{Reduce}
Reduce takes the output of Map, and accumulates the elements in some way.
  \begin{minted}{javascript}
function reduce(fn, a, init) {
  var s = init;
  for (i = 0; i < a.length; i++)
      s = fn(s, a[i]);
  return s;
}
	\end{minted}
	\citeurl{joelonsoftware.com/items/2006/08/01.html}
\end{frame}


\begin{frame}[fragile]{Map Reduce in CouchDB}
Reduce takes the output of Map, and accumulates the elements in some way.
  \begin{minted}{javascript}
function(doc) {
  if(doc.date && doc.title) {
    emit(doc.date, doc.title);
  }
}
	\end{minted}
	
  \begin{minted}{javascript}
function(keys, values, rereduce) {
  if (rereduce)
    return sum(values);
	else
    return values.length;
}
	\end{minted}
	
	\citeurl{http://guide.couchdb.org/draft/views.html}

\end{frame}


\begin{frame}{Security}
  \begin{itemize}
    \item HTTP is not encrypted.
    \item HTTPS is a protocol based on HTTP, but it provides security.
    \item GET and POST are by far the most commonly used HTTP methods (by web developers).
    \item Data sent by GET and POST will be encrypted over HTTPS.
    \item However, it's generally accepted that POST is more secure for sending sensitive data.
    \item This is because browsers will typically cache and servers will typically log URLS, with the data encoded in them.
  \end{itemize}
\end{frame}



\begin{frame}[fragile]{HTTP access control (CORS)}
  \begin{description}
    \item[cross-origin] HTTP requests occur when a resource makes a request from a different domain than it was served from.
    \item[img] tags often do this -- for example, an image might be included from a photo sharing service.
    \item[css and js] files are also often served from domains different from the original.
    \item[Browsers] usually restrict cross-origin requests initiated by scripts.
  \end{description}
  \citeurl{developer.mozilla.org/en-US/docs/Web/HTTP/Access\_control\_CORS}
\end{frame}

\end{document}